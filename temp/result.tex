\usepackage{fullpage}
\usepackage{enumerate}
\usepackage[mathscr]{euscript}
\usepackage{graphicx}
\setlength{\parskip}{0.1in}

\newlength{\toppush}
\setlength{\toppush}{2\headheight}
\setlength{\parindent}{0.0in}
\addtolength{\toppush}{\headsep}

\newtheorem{conjecture}{Conjecture}
\newtheorem{proposition}{Proposition}
\newtheorem{theorem}{Theorem}
\newtheorem{claim}{Claim}
\newtheorem{definition}{Definition}
\newtheorem{note}{Note}
\newtheorem{lemma}{Lemma}
\newtheorem{corollary}{Corollary}
\newenvironment{proof}{{\em Proof:}}{\hfill\rule{2mm}{2mm}}


\def\subjnum{COMP 170}
\def\subjname{Theory of Computation}
\def\doheading#1#2#3{\vfill\eject\vspace*{-\toppush}%
  \vbox{\hbox to\textwidth{{\bf} \subjnum: \subjname \hfil Prof. Lenore  Cowen} %
    \hbox to\textwidth{{\bf} Tufts University \hfil#3\strut}%
    \hrule}}

\newcommand{\htitle}[1]{\vspace*{3.25ex plus 1ex minus .2ex}%
  \begin{center}{\large\bf #1}\end{center}}


\begin{document}

\doheading{2}{title}{Spring 2021}
\htitle{HW 6: due Thursday, April 15 at 9pm  }

\medskip
\subsubsection*{Problem 1}

Consider the following functions

\begin{itemize}
\item $a(n) = 3n^2 + 7$
\item $b(n) = 7 n \log n$
\item $c(n) = 3^n$
\item $d(n) =  2 \sqrt{n} + 400,000$
\item $e(n) = 10 n^2 - 40$
\item $f(n) = \log \log n$
\item $g(n) =  2 n^{2/3}\log^2 n$
\item $h(n) = 2^{\sqrt{n}}$
\item $i(n) = n!$
\item $j(n) = e^n * n^{23}$
\item $k(n) = 2^{2^n}$
\item $l(n) = n^{\log n}$
\end{itemize}

\begin{enumerate}

\item Order these functions from left to right so that $a \leq b$ means $O(a(n))
 \leq O(b(n))$. For example, you might  write

\[ a \leq b \leq c \leq d \leq e \leq f \leq g \leq h \leq i \leq j \leq k \leq
l \]

but of course that's not the right answer. What is?

\item A function is called {\em super-polynomial\/} if it is not $O(n^k)$ for some constant $k$. Which of the functions listed above are {\em super-polynomial}?


\item A function is called {\em sub-exponential\/} if it runs in time
$2^{o(n)}$.  Which of the functions would you
describe as {\em sub-exponential?\/}

\end{enumerate} 


\newpage

\subsubsection*{Problem 2}

Let $S= \{ x_1, \ldots x_n \}$ be a set of $n$ distinct positive integers, and let $t$ also be a positive integer.  Define
\[ \mbox{SUBSET-SUM}  = \{ (S,t) \mid \mbox{ some subset of $S$ sums to exactly
$t$}  \} \]
Show that SUBSET-SUM is in NP.



\newpage

\subsection*{Problem 3}


Suppose we had a Turing Machine $X$ that could decide SUBSET-SUM (from Problem 2) in polynomial time. Explain how to use $X$ to construct a new polynomial time Turing Machine that not only says YES/NO when given an instance $(S,t)$, but also, when the answer is YES, explicitly returns a subset of $S$ that sums to $t$.



\newpage

\subsection*{Problem 4}


A 2-vertex coloring is a function that assigns the colors {\em blue\/} and {\em
red\/} to vertices of a graph. A 2-vertex coloring is called {\em proper\/} if n
o two vertices that are connected by an edge receive the same color.
Let
\[ \mbox{2COLOR } = \{ G \mid G \mbox{ has a proper 2-vertex coloring} \} \]

Show 2COLOR is in NP.



\newpage


\subsection*{Problem 5}

Show 2COLOR (from Problem 4) is in P. 


\end{document} 
